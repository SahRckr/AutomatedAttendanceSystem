\chapter{Introduction}
\section{Overview of Face Based Biometric Systems}
Face recognition has been devoted as one of the most successful and interesting application in image processing. It is also considered to be as old as the computer vision itself because of its non-invasive nature and because it is the primary way of identifying and distinguishing people. Face recognition is gradually evolving into a universal way of person identification since it virtually requires zero effort from the user comparing to the other biometric methods available. Biometric face recognition is basically used in three main domains: time attendance systems and employee management, visitor management systems, and last but not the least authorization systems and access control systems.
The first step in face recognition system is to detect the face in an image. The main objective of face detection is to find whether there are any faces in the image or not. If the face is present, then it returns the location of the image and extent of the each face. Pre-processing is done to remove the noise and reliance on the precise registration.
\begin{figure}[h]
	\centering
	Insert image here /(Figure 1.1/)
	%\includegraphics[height=4cm]{images/vtu.png}
	%\hspace{0.1\textwidth}
\end{figure}
\vfill


\section{Literature Survey}
\begin{itemize}

\item According to \textbf{Ana Bertran, Huanzhou Yu, Paolo Sacchetto}, they have presented a face detection algorithm for colour images that uses colour segmentation, connected component analysis and multi-layer template-matching. Their method uses the colour information in HSV space, compensates for the luminance condition of the image, and overcomes the difficulty of separating faces that are connected together using image morphology processing. Finally, an enhanced version of the template-matching algorithm was used to detect all human faces and reject the non-faces such as hands and clothes. Experimental results have shown that their approach detected 99.4\% of the faces present in their dataset. The only one missing face is due to very dark glasses. No false detection was experienced during the analysis.

\item According to \textbf{Mayank Chauhan, Mukesh Sakle}, they have explained different existing face detection algorithms like Feature based face detection, Geometric based face detection, High-Level language based face detection, Haar like feature based face detection. Then they have compared the both Haar like and Feature based face detection methods with respect to their Precision, Execution time, Learning time and the Ratio between detection rate \& false alarm. Finally they have mentioned the Merits and Demerits of Feature based, Geometric based and Haar like face detection techniques.
\item According to \textbf{Matthias Dantone, Juergen Gall, Juergen Gall, Luc Van Gool}, they had presented a real-time algorithm for facial feature detection based on the novel concept of conditional regression forests. Such ensembles of regression trees estimate the position of several facial landmarks conditional to the probability of some global face properties. In their work, they have demonstrated the benefits of conditional regression forests by modeling the appearance and location of facial feature points conditional to the head pose. The proposed method achieves an accuracy comparable to the performance of human annotators on a large, challenging database of faces captured “in the wild”.
\item According to \textbf{Swarup Ku. Dandpat, Sukadev Meher,} they had presented a new face recognition method using truncated DCT-PCA. It is fast, relatively simple, and has been shown to work well in constrained environment The DCT coefficients were truncated in exponential or Gaussian way giving some weight to the all type of frequencies components. The principal components were selected for each class to reduce the Eigen space. With these eigenvectors the test images were classified based on Euclidean distance. As shown in the results the proposed method had greater accuracy with the other existing methods. This work could be applied to the other databases to see the result of proposed method for cross validation. One factor to look out for was the computational complexity involved there. This would be a major issue when trying to implement the system on real time system. The research would be focused to develop the computational model for face recognition that would be fast, simple and accurate in different environments.
	
\item According to \textbf{M Sharmila Kumari, Swathi Salian,} they study an efficient face recognition technique using combined approaches of discrete cosine transform (DCT) and principal component analysis (PCA). In DCT-PCA, PCA was applied onto the extracted DCT coefficients of the face images. This technique was based on the Fourier spectra of the facial images. The experiments were conducted on the standard face datasets such as ORL, UMIST and real dataset and the performance of the DCT-PCA technique was compared with the existing DCT and PCA. From the experimental results, it was proved that the DCT-PCA based technique for face recognition improved recognition rate than the conventional PCA and DCT.
	
\item According to \textbf{Abhishek Jha,} he had implemented an attendance system for a lecture, section or laboratory by which lecturer or teaching assistant record student’s attendance. It saves time and effort, especially if it is a lecture with huge number of students. The complete system is implemented in MATLAB. This attendance system shows the use of facial recognition techniques for the purpose of student attendance and for the further process this record of student can be used in exam related issues.

\end{itemize}

\section{Problem Statement}
Considering an image representing a frame taken from a video, automatic face recognition is particularly a complex task that involves location of faces in cluttered backgrounds. The human face is also a very challenging pattern to detect unlike the other biometric approaches. This is mainly because even though the anatomy of the face is rigid enough so that all the faces have the same structure, there are a lot of environmental and personal factors that affect facial appearances. The main concern is the large variability of the recorded image due to pose, illumination conditions, facial expressions, use of cosmetics, different hairstyle, presence of glasses, beard etc. Image of the same person taken at different times, may also exhibit more variability.

\section{Motivation of the problem}
Every time a lecture, section or laboratory starts the lecturer or teaching assistant delays the lecture to record student’s attendance. This is a lengthy process and takes lot of time and effort, especially if it is a lecture with huge number of students. It also causes a lot of disturbance and interruption when an exam is held. Moreover the attendance sheet is subjected to damage and loss while being passed on between different students of teaching staff.
When the number of students enrolled in a certain course is huge, the lecturers tend to call a couple of students name at random which is not fair student evaluation process either. Finally, these attendance records are used by the staff to monitor the student’s attendance rates. This process could be easy and effective with a small number of students but on the other hand dealing with the records of a large number of students often leads to human error.

\section{Proposed System}
We propose solutions to all the above mentioned problems by providing an automated attendance system for all the students that attend a certain lecture, section, laboratory or exam at its specific time, thus saving time, effort and reducing distractions and disturbance. Another advantage concerning exams, is when the lecturer or the advisor accidentally losses an exam paper or the student lies about attending the exam, there will be a record of the students’ attendance for the exam at that time, thus protecting both lecturers and students rights. In addition, an automated performance evaluation would provide more accurate and reliable results avoiding human error.
The main objective of the system is to provide an automated attendance system that is practical, reliable and eliminates disturbance and time loss in traditional attendance systems. A further objective is to present a system that can accurately evaluate students’ performances depending on their recorded attendance rate.
This project presents an appearance based face recognition which is based on the DCT (Discrete Cosine Transform) approach. DCT based techniques extracts most significant DCT coefficients of the training images as a part of feature extraction. The DCT can either be applied on entire frontal face image to obtain global DCT features or the local features like nose, eyes etc. can be extracted manually and then DCT can be applied onto them to form local DCT features. When local and global features are combined, DCT gives a relatively high recognition rate.

Insert figure 5.2
\section{Applications}
Since the driving force of this attendance machine is the practicality of face recognition, we shall consider some of the applications of face recognition and list it down here.
